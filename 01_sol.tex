\documentclass[11pt]{article}

% File configuration packages
\usepackage[a4paper, total={190mm, 270mm}]{geometry}
\usepackage{fancyhdr}
\usepackage{setspace}
\usepackage[spanish]{babel}
\usepackage{booktabs} % Quality tables
\usepackage{apacite}
\usepackage{array}
\usepackage{float} % Arrange for floating objects
\usepackage{enumerate}
\usepackage[
  colorlinks=true,
  linkcolor=black,
  anchorcolor=black,
  citecolor=black,
  filecolor=black,
  menucolor=black,
  runcolor=black,
  urlcolor=black
]{hyperref}


% Math related packages
\usepackage{amsthm}
\usepackage{amsmath}
\usepackage{amssymb}
\usepackage{amsfonts}
\usepackage{mathabx}
\usepackage{mathstyle}
\usepackage{mathtools}
\newcommand*\rfrac[2]{{}^{#1}\!/_{#2}} % Split frac: #1/#2
\newtheorem{teorema}{Teorema}
\newtheorem{lemma}{Lemma}
\newtheorem{corolario}{Corolario}
\newtheorem{contra}{Contraejemplo}
\renewcommand\qedsymbol{$\blacksquare$}

% Graphs and Images related packages
\usepackage{graphicx}
\graphicspath{ {./} }
\usepackage{pgfplots}
\usepackage{tikz}
\usetikzlibrary{positioning}

\title{ Tópicos de Análisis Real: Ejercitación 1 }

\date{ Segundo semestre 2020 }


\begin{document}
\maketitle
\begin{enumerate}
  \item Decidir si los siguientes conjuntos son o no subespacios.
    \begin{enumerate}
      \item $S_{1}=\left\{x \in \mathbb{R}^{2} / \ x_{1}+x_{2}=5\right\}$\\
        No es subespacio porque $(0,0) \notin S_1$ porque $0+0\neq 5$
      \item $S_{2}=\left\{x \in \mathbb{R}^{3} / \ 3 x_{1}-x_{3}=0\right.$ y 
        $\left.x_{3}+x_{2}=0\right\}$\\
        Veamos, un elemento $x$ de $S_2$ tiene la siguiente forma 
        $x = v \left( \frac{1}{3} ,-1,1 \right) \forall v \in \mathbb{R}$, 
        tenemos entonces la forma paremétrica de una recta con dirección 
        $(\frac{1}{3},-1,1)$ entonces $S_2$ es una recta en $\mathbb{R}^3$ 
        que pasa por el origen, por ende es un subespacio de $\mathbb{R}^3$
      \item $S_{3}=\left\{x \in \mathbb{R}^{3} / \ x=\alpha(1,2,3),\right.$ con 
        $\left.\alpha \in \mathbb{R}\right\}$\\
        Sabemos que $S_3$ es una recta en $\mathbb{R}^3$ que pasa por el origen,
        por ende sera un subespacio de $\mathbb{R}^3$
      \item $S_{4}=\left\{x \in \mathbb{R}^{3} / \ x_{1}=x_{2} x_{3}\right\}$\\
        Sea $v = (v_2 v_3, v_2, v_3) \in S_4$ y $\alpha \in \mathbb{R}$,
        \[ \alpha v =  (\alpha v_2 v_3,\alpha v_2, \alpha v_3) \]
        Como la primera coordena no respeta $x_1 = x_2 x_3 $ $\alpha v \notin S_4$
        entonces $S_4$ no es un subespacio de $\mathbb{R}^3$.
      \item $S_{5}=\left\{x \in \mathbb{R}^{4} / \ \langle x, a\rangle =0 \right\}$,
        donde $a \in \mathbb{R}^{4}$\\
        Sea $v, w \in S_5$ \\
        \begin{itemize}
          \item $ \langle 0, \alpha \rangle = 0 \implies 0 \in S_5$
          \item $ \langle v + w, \alpha \rangle = \langle v, \alpha \rangle 
            + \langle w, \alpha \rangle = 0 + 0 = 0 \implies v + w \in S_5 $
          \item Sea $r$ in $\mathbb{R}$.\\
           $ \langle rv , \alpha \rangle = r \langle v, \alpha \rangle = r 0 = 0 
           \implies rv \in S_5$  
        \end{itemize}
        Como se cumplen estas tres propiedades podemos afirmar que $S_5$ es un 
        subespacio de $\mathbb{R}^4$
    \end{enumerate}
  \item Sea $W=\left[ (1,-1, 1), (0,0,1) \right]$\\
    $(2,2,1) \notin W \because \ \nexists \ \alpha, \beta / 
    \ (2,2,1) = \alpha (1, -1, 1) + \beta (0, 0, 1)$ \\
    $(5,-5,2) = 5 (1, -1, 1) + (-3) (0, 0, 1) \implies (5, -5, 2) \in W$\\

    Sea $S=\left\{ x \in \mathbb{R}^4 / x_1 + x_2\right\}$. Un conjunto de vectores 
    puede generar a $S$  si pertenecen a $S$ y son linealmente independientes.
    \begin{itemize}
      \item $(1,-1,1);(2,-2,2)$ \\
        No generan $S$ porque $2 (1,-1,1) = (2, -2,2)$, es decir, no son linealmente
        independientes.
      \item $(1,1,0) ;(1,-1,0)$ \\
        No generan $S$ porque $(1,1,0)\notin S$
      \item $(1,-1,0);(1,-1,1)$ \\
        Si generan a $S$ porque son linealmente independientes y ambos vectores
        están en $S$.
    \end{itemize}
  \item \begin{enumerate}
          \item Dados $S=\left\{x \in \mathbb{R}^{3}: 3 x_{1}+2 x_{2}+x_{3}=0\right\}$
            $T=[(1,1,2) ;(0,1,-1)]$ $W=[(3,1,0) ;(2,2,-1)]$
            \begin{itemize}
              \item $S\cap T = [(1,-6,9)]$
                \begin{proof}
                  \[ 3 x_1 + 2 x_2 + x_1 = 0 \tag{Condicion para pertenecer a $S$}\]
                  \[ \lambda_1 (1,1,2) + \lambda_2 (0,1,-1) = (\lambda_1, \lambda_1
                    + \lambda_2, 2 \lambda_1 - \lambda_2) 
                  \tag{Condicion para pertenecer a $T$} \]
                  Para $x \in S \cap T$ se deben cumplir ambas condiciones,
                  si unimos las dos condiciones en una ecuación obtenemos,
                  \[ 3 \lambda_1 + 2 (\lambda_1+\lambda_2) + 
                  2\lambda_1 - \lambda_2 = 0\]
                  Resolviendo la ecuación anterior tenemos, $-7\lambda_1 = \lambda_2$.
                  Usando esto y la condición para pertenecer a $T$ obtenemos 
                  \[ \lambda_1 (1, -6, 9) \]
                \end{proof}  
              \item $T \cap W = \left[\left(\frac{1}{8},\frac{11}{8},-1\right)\right] $
                \begin{proof}
                  \[ \lambda_1 (1,1,2) + \lambda_2 (0,1,-1) = (\lambda_1, \lambda_1
                    + \lambda_2, 2 \lambda_1 - \lambda_2) 
                  \tag{Condicion para pertenecer a $T$} \]
                  \[ \lambda_3 (3,1,0) + \lambda_4 (2,2,-1) = (3\lambda_3 + 2\lambda_4
                      , \lambda_3  + 2 \lambda_4, - \lambda_4) 
                  \tag{Condicion para pertenecer a $W$} \]
                  Para $x \in W \cap T$ se deben cumplir ambas condiciones,
                  si unimos las dos condiciones obtenemos,
                  \[ (\lambda_1, \lambda_1 + \lambda_2, 2 \lambda_1 - \lambda_2) =
                  (\lambda_3 + 2\lambda_4, \lambda_3  + 2 \lambda_4, - \lambda_4) \] 
                  Resolvemos el sistema de 3 ecuaciones y 4 incógnitas y obtenemos,
                  \[ \begin{cases}
                      \lambda_1 =& \frac{1}{8} \lambda_4 \\
                      \lambda_2 =& \frac{5}{4} \lambda_4 \\
                      \lambda_3 =& \frac{-5}{8} \lambda_4 \\
                      \lambda_4 & variable \ libre 
                  \end{cases} \]
                  Usando el resultado anterior y cualquiera de las dos condiciones 
                  de pertenencia obtenemos,
                  \[ \lambda_4 \left(\frac{1}{8}, \frac{11}{8}, -1 \right) \]
                \end{proof}
              \item $S \cap[(3,2,1)] = \left\{ 0 \right\}$
                \begin{proof}
                  \[ 3 x_1 + 2 x_2 + x_1 = 0 \tag{Condicion para pertenecer a $S$}\]
                  \[ \lambda_1 (3,2,1) = (3\lambda_1, 2\lambda_1, \lambda_1) 
                  \tag{Condicion para pertenecer a $[(3,2,1)]$} \]
                  Para $x \in S \cap [(3,2,1)]$ se deben cumplir ambas condiciones,
                  si unimos las dos condiciones en una ecuación obtenemos,
                  \[ 9 \lambda_1 + 4 \lambda_1 + \lambda_1 = 0\]
                  Resolviendo la ecuación anterior tenemos, $\lambda_1 = 0$.
                  Esto nos quiere decir que el único vector en $ S \cap [(3,2,1)]$ 
                  es el $(0,0,0)$.\\
                \end{proof}  
                Una explicación intuitiva de porque la intersección es solo el vector
                nulo, es porque la intersección entre una recta y un plano en 
                $\mathbb{R}$ puede ser 
                \begin{itemize}
                  \item La recta, si esta esta incluida en el plano
                  \item El elemento nulo, en otro caso
                \end{itemize}
                y como en este caso el vector que genera la recta no esta en el plano, 
                la recta tampoco esta, por ende la intersección sera el elemento nulo.
            \end{itemize}    
          \item Sea $\alpha \in \mathbb{R}^n$, $V$ y $H$ subespacios de 
            $\mathbb{R}^n$ talque
            $V = [\alpha]$\\
            $H=\left\{x \in \mathbb{R}^{n} / \langle \alpha, x \rangle=0\right\}$ \\
            \begin{proof} Quiero ver que $V \cap H = \{ 0 \}$ \\
              Para que $x \in V \cap H$ se deben cumplir dos cosas
              \begin{itemize}
                \item $x = k \alpha$ para algún $k \in \mathbb{R}$
                \item $\langle x, \alpha \rangle = 0 $
                \end{itemize}
                Si juntamos ambas condiciones obtenemos 
                \[ \langle k \alpha, \alpha \rangle = k \langle \alpha, \alpha \rangle
                = k ||\alpha||^{\rfrac{1}{2}} = 0\]
                La expresión anterior solo se cumple en dos casos,
                \begin{itemize}
                  \item Si $k=0$ en este caso $x$ que pertenece a $V \cap H$ debe ser
                    el elemento nulo.
                  \item Si $\alpha = (0, \dots 0)$, en este caso $V = \{0\}$ entonces 
                    la intersección de $V$ con cualquier otro caso, solo sera el
                    elemento nulo, en especial la intersección con $H$.
                \end{itemize}
            \end{proof}
        \end{enumerate}
      \item Sea $S=\left[\left(1,1,\frac{1}{3}\right)\right]$ y $T=[(2,1,1)]$\\
        Notemos que tanto $S$ como $T$ son rectas en $\mathbb{R}$ entonces $S+T$ va a
        ser un plano con direcciones $\left(1,1,\frac{1}{3}\right)$ y $(2,1,1)$
        entonces el espacio $S+T$ es $\left[\left(1,1,\frac{1}{3}\right);(2,1,1)\right]$
        \begin{proof}
          Sean $\alpha, \beta \in \mathbb{R}$,\\
          $s = \alpha \left(1,1, \frac{1}{3} \right)\in S$\\
          $t = \beta (2, 1, 1) \in T$\\
          Entonces veamos $s+t \in S+ T$
          \[ s + t = \alpha \left( 1, 1, \frac{1}{3} \right) + \beta (2,1,1)\]
          Un elemento cualquiera de $S+T$ lo escribimos como una combinación lineal
          esos dos vectores entonces $\left(0,1,\frac{1}{3}\right)$ y $(2,1,1)$ es un 
          sistema generador de $S+T$.\\
        \end{proof}
        Podemos escribir $(4,5,-3)$ de la siguiente manera
        \[ (4,5,-3) = 6\left(1,1,\frac{1}{3}\right) -(2,1,1) \]
        El vector $(1, 3, 0)$ no se puede escribir de esta manera porque $\nexists \
        \alpha \beta /$
        \[ (1,3,0) = \alpha \left(1,1,\frac{1}{3}\right) + \beta (2,1,1) \]
    \item Sean $S, T, W$ subespacios de $\mathbb{R}$,\\
      $S=\{x \in \mathbb{R}^3 /x=\alpha (1,1,2)\}$ y $s=(\sigma,\sigma,2\sigma)\in S$\\
      $T=\{x \in \mathbb{R}^3 / x_1-x_3=0\}$ y $t = (\tau_1, \tau_2, \tau_1) \in T$\\
      $W=\{x \in \mathbb{R}^3 / x_1=0\}$ y $w = (0, \omega_2, \omega_3) \in W$\\
      $W+T = \mathbb{R}^3$ Veamos,
      \[ w + t = (\tau_1, tau_2 + \omega_2, \tau_1 + \omega_3) = (x_1, x_2, x_3)\]
      Como tenemos un sistema de 4 incógnitas y 3 ecuaciones, nos queda una variable
      libre ($\tau_2$ o $\omega_2$) podemos escribir cualquier elemento 
      de $\mathbb{R}$ (de infinitas formas diferentes) como la suma de un 
      elemento en $W$ y otro en $T$.\\
      $S+T = \mathbb{R}^3$ Veamos,
      \[ s+t = (\alpha + \tau_1, \alpha + tau_2, 2 \alpha \tau_1) = (x_1, x_2, x_3)\]
      Como tenemos un sistema de 3 incógnitas y 3 ecuaciones, nos queda una única
      solución, entonces podemos escribir cualquier elemento de $\mathbb{R}$
      de una única forma como la suma de un elemento en $S$ y otro en $T$.
      En este caso vale la suma directa.
  \item Sea $\alpha \in \mathbb{R}^n$, $V$ y $H$ subespacios de 
    $\mathbb{R}^n$ talque
    $V = [\alpha]$\\
    $H=\left\{x \in \mathbb{R}^{n} / \langle \alpha, x \rangle=0\right\}$ \\
    \begin{proof} Quiero ver que $V \oplus H = \mathbb{R}^n$.\\
      Por la definición de \textit{suma directa} dada en clase;
        \begin{center}
          \framebox{\begin{minipage}{20em}\begin{center}
            Sean $V, H$ subespacios de $\mathbb{R}^n$ y $V \cap H = \{0\}$
            $$ V + H = \mathbb{R}^n \implies V \oplus H = \mathbb{R}^n $$
          \end{center} \end{minipage}}
        \end{center}
	  Como en el ejercicio \textbf{3} probamos que $V \cap H = \{0\}$, 
      entonces por la definición de suma directa 
      \[ V \oplus H = \mathbb{R}^n \]
    \end{proof}
  \item $|a \cos \theta+b \sin \theta|^{2} \leq a^{2}+b^{2}$         
    \begin{proof} Sean $x, y \in \mathbb{R}^2$.
      $x = (a, b)$ y
      $y = (\cos \theta, \sin \theta)$\\
      Partimos de la desigualdad de Cauchy Shwartz.
      \[\begin{aligned}
      |\langle x, y \rangle | &\leq ||x|| \ ||y|| \\
      |a \cos \theta + b \sin \theta| &\leq \sqrt{a^2 + b^2} 
      \sqrt{\cos^2 \theta + \sin^2 \theta}\\
      (a \cos \theta + b \sin \theta)^2 &\leq a^2 + b^2\\
      \end{aligned}\]
    \end{proof}
  \begin{proof}
    \[ \begin{aligned}
      a^2 + b^2 &=  a^2 + b^2 \\
      &= a^2 (\cos^2 \theta + \sin^2 \theta) + b^2 (\cos^2 \theta + \sin^2 \theta)\\
      &= a^2 \cos^2 \theta + a^2 \sin^2 \theta +b^2 \sin^2 \theta + b^2 \cos^2 \theta \\
      &= a^2 \cos^2 \theta + b^2 \sin^2 \theta  + a^2 \sin^2 \theta +b^2 \cos^2
              \theta+ab \cos \theta \sin \theta-ab\cos\theta \sin \theta\\              
      &= (a^2 \cos^2 \theta + b^2 \sin^2 \theta + ab \cos \theta \sin \theta )
                + (a^2 \sin^2 \theta + b^2 \cos^2 \theta - ab \cos \theta \sin \theta)\\
      &= (a \cos \theta + b \sin \theta)^2 + (a \sin \theta - b \cos \theta)^2\\
      &\geq  (a \cos \theta + b \sin \theta)^2 \\
      &\geq  |a \cos \theta + b \sin \theta|^2 \\
    \end{aligned} \]
  \end{proof}
  \item Sean $v , u \in \mathbb{R}^n$ talque $\langle u, v \rangle = 0$
    \[ || u-v ||^2 = ||u||^2 + ||v||^2 \]
    \begin{proof}
      \[ \begin{aligned} 
        ||u - v ||^2 &= \langle u - v, u - v \rangle \\
                     &= \langle u,u\rangle + \langle v,v \rangle- 2\langle u,v\rangle \\
                     &= || u ||^2 + ||v||^2 \\
      \end{aligned}\]
    \end{proof}
    Entonces si, $||v||=||u||= 1$
    \[ ||u-v|| = \sqrt{2} \]
  \item $\|u+v\|^{2}+\|u-v\|^{2}=2\|u\|^{2}+2\|v\|^{2}$
    \begin{proof}
      \[ \begin{aligned}
          ||u+v||^2 + ||u-v||^2 &=\langle u+v,u+v \rangle + \langle u-v,u-v \rangle\\   
           &= \langle u,u \rangle + \langle v,v \rangle + 2 \langle u,v \rangle +
            \langle u,u \rangle + \langle v,v \rangle - 2 \langle u,v \rangle \\
           &= || u ||^2 + ||v||^2 + || u ||^2 + ||v||^2 \\
           &= 2 || u ||^2 + 2 ||v||^2 \\
      \end{aligned} \]    
    \end{proof}  
  \item $[(2,1)]^{\perp} = [(1,-2)$]\\
    \begin{itemize}
    \item $(6,3) = 3 (2,1) + 0 (1,-2)$
    \begin{center}
      \begin{tikzpicture}[scale=0.5]
        % Axis
        \draw [thick](-3.25,0) -- (6.5,0);
        \draw [thick](0,-1) -- (0,6.5);
        \node [left] at (7.5,0) {$x$};
        \node [below] at (0,7.5) {$y$};
        %Subespace and orthogonal complement
        \draw [thick, blue] (-2,-1) -- (6.5,3.25);
        \draw [thick, blue] (0.5,-1) -- (-3.25,6.5);
        \draw [fill] (6,3) circle [radius =0.06];
        \node [below] at (5,4) {$(6,3$)};
      \end{tikzpicture}
    \end{center}
    \item $(-2,4) = 0 (2,1) + (-2) (1,-2)$
    \begin{center}
      \begin{tikzpicture}[scale=0.5]
        % Axis
        \draw [thick](-3.25,0) -- (6.5,0);
        \draw [thick](0,-1) -- (0,6.5);
        \node [left] at (7.5,0) {$x$};
        \node [below] at (0,7.5) {$y$};
        %Subespace and orthogonal complement
        \draw [thick, blue] (-2,-1) -- (6.5,3.25);
        \draw [thick, blue] (0.5,-1) -- (-3.25,6.5);
        \draw [fill] (-2,4) circle [radius =0.06];
        \node [below] at (-1.3, 6.2) {$(-2,4)$};
      \end{tikzpicture}
    \end{center}
    \item $(5,6) = \frac{16}{5} (2,1) - \frac{7}{5} (1,-2)$
    \begin{center}
      \begin{tikzpicture}[scale=0.5]
        % Axis
        \draw [thick](-3.25,0) -- (6.5,0);
        \draw [thick](0,-1) -- (0,6.5);
        \node [left] at (7.5,0) {$x$};
        \node [below] at (0,7.5) {$y$};
        %Subespace and orthogonal complement
        \draw [thick, blue] (-2,-1) -- (6.5,3.25);
        \draw [thick, blue] (0.5,-1) -- (-3.25,6.5);
        \draw [fill] (5,6) circle [radius =0.06];
        \node [left] at (7.3,6) {$(5,6)$};
        \draw [thick, dashed, red] (6.4,3.2) -- (5,6);
        \draw [thick, dashed, red] (-1.4,2.8) -- (5,6);
      \end{tikzpicture}
    \end{center}
    \end{itemize}
  \item \begin{enumerate}
      \item $S = [(1,-2)]$ y $S^{\perp} = [(2,1)]$
        \begin{center}
          \begin{tikzpicture}[scale=0.5]
            % Axis
            \draw [thick](-3.25,0) -- (6.5,0);
            \draw [thick](0,-1) -- (0,6.5);
            \node [left] at (7.5,0) {$x$};
            \node [below] at (0,7.5) {$y$};
            %Subespace and orthogonal complement
            \draw [thick, blue] (-2,-1) -- (6.5,3.25);
            \draw [thick, blue] (0.5,-1) -- (-3.25,6.5);
            \node [left] at (-1,4) {$S$};
            \node [left] at (3.7,1) {$S^{\perp}$};
          \end{tikzpicture}
       \end{center}
       $(1,1) = \rfrac{3}{5} (2,1) - \rfrac{1}{5} (1,-2)$\\
       $(4,5) = \rfrac{13}{5} (2,1) - \rfrac{6}{5} (1,-2)$
      \item $S = [(1,-1)]$ y $S^{\perp} = [(1,1)]$
        \begin{center}
          \begin{tikzpicture}[scale=0.5]
            % Axis
            \draw [thick](-3.25,0) -- (6.5,0);
            \draw [thick](0,-1) -- (0,6.5);
            \node [left] at (7.5,0) {$x$};
            \node [below] at (0,7.5) {$y$};
            %Subespace and orthogonal complement
            \draw [thick, blue] (-1,-1) -- (6.5,6.5);
            \draw [thick, blue] (1,-1) -- (-3.25,3.25);
            \node [left] at (-1,2.2) {$S$};
            \node [left] at (3.7,2) {$S^{\perp}$};
          \end{tikzpicture}
       \end{center}
       $(1,1) = 0 (1,-1) + 1 (1,1)$\\
       $(4,5) = -\rfrac{1}{2} (1,-1) + \rfrac{9}{2} (1,1)$
     \item $S=\left\{x \in \mathbb{R}^3 / x_1+x_2-x_3=0\right\} = 
       [(1,-1,0);(1,0,1)]$ y $S^{\perp} = [(-1,-1,1)]$\\
       El vector para generar $S^{\perp}$, que en este caso es una recta, lo obtengo
       haciendo el producto vectorial entre dos vectores linealmente independientes,
       en este caso lo obtuve haciendo $(1,-1,0) \times (1,0,1) = (-1,-1,1)$.\\
       $(1, 1, -1)= 0 (1,-1,0) + 0 (1,0,1) -1 (-1,-1,1) $\\
       $(2, -1, 3)= \rfrac{1}{3}(1,-1,0)+\rfrac{2}{3}(1,0,1)+\rfrac{2}{3}(-1,-1,1) $\\
     \item $S=\left\{x \in \mathbb{R}^3 / 2 x_1-x_2+3x_3=0\right\} = 
       [(1,2,0);(3,0,-2)]$ y $S^{\perp} = [(-4,2,-6)]$\\
       $ (1, 1, -1) = \rfrac{3}{7}(1,2,0) +\rfrac{2}{7}(3,0,-2) +\rfrac{1}{14}(-4,2,-6)$
       $ (2, -1, 3) = 0 (1,2,0) + 0 (3,0,-2) - \rfrac{1}{2} (-4,2,-6)$
      \end{enumerate}
    \item $S=[(1,1,-1);(0,2,2)]$, $S^{\perp}=[(4,-2,2)]$ y 
      $T=\{x \in \mathbb{R}^3 /-x_1+kx_2+k^2x_3=0\}$.\\
      Primero veamos las dimensiones de los distintos espacios, $S$ y $T$ son planos y 
      $S^{\perp}$ es una recta, entonces si puede darse el caso que $T$ contenga a 
      $S^{\perp}$. Como tanto $S^{\perp}$ y $T$ son subespacios, alcanza con que
      compartan un punto para que $S^{\perp} \subset T$.\\
      Si tomamos el vector generador de $S^{\perp}$ y la ecuación del plano $T$ tenemos,
      \[ -4 -2 k + 2 k^2 = 0\]
      La ecuación anterior tiene dos soluciones en $\mathbb{R}$, $k=-1$ o $k=2$.
\end{enumerate}
\end{document}

